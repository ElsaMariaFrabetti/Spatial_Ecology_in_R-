\documentclass{beamer} % class for presentation
\usepackage{graphicx} % Required for inserting images

\usetheme{Frankfurt} % presentaion style
\usecolortheme{beaver} % color

\title{Spatial Ecology in R}
\author{Elsa Frabetti}
\institute{Global change ecology - Unibo}
\date{June 2024}


\begin{document}

\maketitle % it writes down the things we insert before begin
\AtBeginSection[]
{
\begin{frame}{Outline}
\tableofcontents[currentsection]
\end{frame}
} % it shows me the outline for every new section


\section{Materials and methods} % 1st section

\begin{frame}{Images} % title
    Images downloaded from \textbf{\href{https://browser.dataspace.copernicus.eu}{Copernicus Browser}}:
    \bigskip
    \begin{itemize}
        \item Searched for the study area.
        \item  Cloud cover lower than 5 \%.
        \item Downloaded images True Color and False Color.
        \item Images format .jpg high resolution.
    \end{itemize}
\end{frame}

\begin{frame}{Packages} 
    For my script I used the following packages:
    \bigskip % to leave space
    \begin{itemize} % bulleted list
        \item terra
        \item imageRy
        \item viridis
        \item ggplot2
        \item patchwork
        \end{itemize}
\end{frame}



\section{Studied area} % 2nd section

\begin{frame}{Where?}
We are in Sardinia, in the area of Montiferru, province of Oristano.
\bigskip
\begin{columns} % making many columns

    \begin{column}{0.5\textwidth} % first column
    \begin{figure}
        \centering
        \includegraphics[width=1\linewidth]{screen sardegna small.png}
        \caption{Province of Scano di Montiferro, OR}
    \end{figure}
    \end{column}
    
    \begin{column}{0.5\textwidth}
    \begin{figure}
        \centering
        \includegraphics[width=1\linewidth]{screen sardegna big.png}
        \caption{Sardinia region}
    \end{figure}
    \end{column}
    
\end{columns}
\end{frame}

\begin{frame}{What happened?}
During July 2021, a massive wildfire destroyed the area of Montiferru, burning almost 13.000 hectares of land, of which 4.000 were woods.\\It seems that the fire was caused accidentally, because a car caught fire on the road, but it turned out to be the worst wildfire happened in Sardinia in the past 24 years. 

\begin{columns}
    \begin{column}{0.5\textwidth}
        \begin{figure}
            \centering
            \includegraphics[width=0.8\linewidth]{pre-incendio tc.jpg}
        \end{figure}
    \end{column}

    \begin{column}{0.5\textwidth}
    \begin{figure}
        \centering
        \includegraphics[width=0.8\linewidth]{post-incendio tc.jpg}
    \end{figure}
    \end{column}
\end{columns}
\end{frame}



\section{Data analysis} % 3rd section

\begin{frame}{Comparison July - August 2021}

\centering
Let's see the difference before and after the fire with true and false color images. 
\begin{figure}
    \centering
    \includegraphics[width=0.5\linewidth]{1plot.png}
\end{figure}
\end{frame}

\begin{frame}{Spectral indices}
\centering % centering the text
Through false color images we can calculate spectral indices for vegetation.
\bigskip % leaving more space
\bigskip
\begin{columns}
    \begin{column}[t]{0.55\textwidth}
    \centering
    \textbf{DVI}\\ \small Difference Vegetation Index
    \begin{equation*}
        DVI = NIR - RED
    \end{equation*}
    \end{column}

    \begin{column}[t]{0.6\textwidth}
    \centering
    \textbf{NDVI}\\ \small Normalized Difference Vegetation Index
    \begin{equation*}
        NDVI = \frac{NIR - RED}{NIR + RED}
    \end{equation*}
    \end{column}   
\end{columns}    
\end{frame}

\begin{frame}{NDVI}
\centering
Let's see NDVI's comparison before and after fire\\
(July - August 2021)
\begin{figure}
    \centering
    \includegraphics[width=1\linewidth]{2plot (2).png}
\end{figure}   
\end{frame}

\begin{frame}{Classification}
\begin{columns}
    \begin{column}{0.65\textwidth}
    \begin{figure}
           \centering
           \includegraphics[width=1\linewidth]{3plot.png}
       \end{figure}   
    \end{column}

    \begin{column}{0.5\textwidth}
       Classification based on NDVI values before and after fire. 
    \end{column}
\end{columns}    
\end{frame}

\begin{frame}{Classification}
\begin{columns}
    \begin{column}{0.65\textwidth}
    \begin{figure}
           \centering
           \includegraphics[width=1\linewidth]{3plot.png}
       \end{figure}   
    \end{column}

    \begin{column}{0.5\textwidth}
    \textbf{July 2021}:
        \begin{itemize}
            \item Vegetation = 51 \%
            \item Bare soil = 49 \%
        \end{itemize}
        \bigskip
        \bigskip
        \bigskip
        \textbf{August 2021}:
        \begin{itemize}
            \item Vegetation = 35 \%
            \item Bare soil = 65 \%
        \end{itemize}
    \end{column}
\end{columns}    
\end{frame}

\begin{frame}{Classification}
\centering
Let's visualize graphically the classes' percentages before and after fire. 
\begin{columns}
    \begin{column}{0.65\textwidth}
    \begin{figure}
        \centering
        \includegraphics[width=1\linewidth]{4plot.png} 
    \end{figure}
    \end{column}

    \begin{column}{0.5\textwidth}
        We can see a clear vegetation reduction after the fire. 
    \end{column}
\end{columns}     
\end{frame}

\begin{frame}{Spectral variability}
\centering
\textbf{PCA} on NIR, R and G bands.\\
\textbf{Moving window} (MW) of different sizes on \textbf{PC1}.
\begin{figure}
      \centering
      \includegraphics[width=0.7\linewidth]{5plot.png}
  \end{figure}      
\end{frame}

\begin{frame}{Spectral variability}
\centering
We can see that the fire reduced the habitat heterogeneity,\\ with a likely consequent decrease of biodiversity.  
\begin{figure}
    \centering
    \includegraphics[width=0.7\linewidth]{5plot.png}
\end{figure}
\end{frame}

\begin{frame}{Recovery}
\centering
Let's compare the area status before fire with the actual situation (August 2023). 
\begin{figure}
       \centering
       \includegraphics[width=0.8\linewidth]{6plot.png}
   \end{figure}   
\end{frame}

\begin{frame}{Recovery}
\centering
Let's do another comparison by calculating NDVI differences between August 2021 and August 2023.
\begin{figure}
     \centering
     \includegraphics[width=0.9\linewidth]{7plot.png}
 \end{figure} 
\end{frame}

\begin{frame}{Recovery}
\centering
We can clearly see that two years after the fire\\ secondary succession is restoring the environment.
\begin{figure}
     \centering
     \includegraphics[width=0.9\linewidth]{7plot.png}
 \end{figure} 
\end{frame}


\section{Conclusion} % 4th section

\begin{frame}{Resume}
\begin{itemize}
    \item Clear vegetation reduction after the fire.
    \bigskip
    \pause \item From July to August 2021, we are witnessing a vegetation loss of the 16\%. 
    \bigskip
    \pause \item After the fire, the habitat resulted to be less heterogeneous with also a decreased biodiversity. 
    \bigskip
    \pause \item Looking at the environment status two years after the fire, NDVI differences tell us that secondary succession is restoring the area.
\end{itemize}
\end{frame}

\begin{frame}{The End}
\centering
\large \textbf{Thanks for the attention!}
\end{frame}



\end{document}
