\documentclass{beamer}
\usepackage{graphicx} % Required for inserting images

\usetheme{Frankfurt}
\usecolortheme{beaver}

\title{Spatial Ecology in R}
\author{Elsa Frabetti}
\institute{Global change ecology - Unibo}
\date{June 2024}


\begin{document}

\maketitle
\AtBeginSection[]
{
\begin{frame}{Outline}
\tableofcontents[currentsection]
\end{frame}
} % ad ogni sezione nuova mi fa vedere l'outline


\section{Materials and methods}

\section{Studied area}

\section{Data analysis}

\begin{frame}{Comparison July - August 2021}

\centering
Let's see the difference before and after the fire with true and false color images. 
\begin{figure}
    \centering
    \includegraphics[width=0.5\linewidth]{1plot.png}
\end{figure}
\end{frame}

\begin{frame}{Spectral indices}
\centering %per centrare il testo
Through false color images we can calculate spectral indices for vegetation.
\bigskip %per lasciare più spazio
\bigskip
\begin{columns}
    \begin{column}[t]{0.55\textwidth}
    \centering
    \textbf{DVI}\\ \small Difference Vegetation Index
    \begin{equation*}
        DVI = NIR - RED
    \end{equation*}
    \end{column}

    \begin{column}[t]{0.6\textwidth}
    \centering
    \textbf{NDVI}\\ \small Normalized Difference Vegetation Index
    \begin{equation*}
        NDVI = \frac{NIR - RED}{NIR + RED}
    \end{equation*}
    \end{column}   
\end{columns}    
\end{frame}

\begin{frame}{NDVI}
\centering
Let's see NDVI's comparison before and after fire\\
(July - August 2021)
\begin{figure}
    \centering
    \includegraphics[width=1\linewidth]{2plot (2).png}
\end{figure}   
\end{frame}

\begin{frame}{Classification}
\begin{columns}
    \begin{column}{0.65\textwidth}
    \begin{figure}
           \centering
           \includegraphics[width=1\linewidth]{3plot.png}
       \end{figure}   
    \end{column}

    \begin{column}{0.5\textwidth}
       Classification based on NDVI values before and after fire. 
    \end{column}
\end{columns}    
\end{frame}

\begin{frame}{Classification}
\begin{columns}
    \begin{column}{0.65\textwidth}
    \begin{figure}
           \centering
           \includegraphics[width=1\linewidth]{3plot.png}
       \end{figure}   
    \end{column}

    \begin{column}{0.5\textwidth}
    \textbf{July 2021}:
        \begin{itemize}
            \item Vegetation = 51 \%
            \item Bare soil = 49 \%
        \end{itemize}
        \bigskip
        \bigskip
        \bigskip
        \textbf{August 2021}:
        \begin{itemize}
            \item Vegetation = 35 \%
            \item Bare soil = 65 \%
        \end{itemize}
    \end{column}
\end{columns}    
\end{frame}

\begin{frame}{Classification}
\centering
Let's visualize graphically the classes' percentages before and after fire. 
\begin{columns}
    \begin{column}{0.65\textwidth}
    \begin{figure}
        \centering
        \includegraphics[width=1\linewidth]{4plot.png} 
    \end{figure}
    \end{column}

    \begin{column}{0.5\textwidth}
        We can see a clear vegetation reduction after the fire. 
    \end{column}
\end{columns}     
\end{frame}

\begin{frame}{Spectral variability}
\centering
\textbf{PCA} on NIR, R and G bands.\\
\textbf{Moving window} (MW) of different sizes on \textbf{PC1}.
\begin{figure}
      \centering
      \includegraphics[width=0.7\linewidth]{5plot.png}
  \end{figure}      
\end{frame}

\begin{frame}{Spectral variability}
\centering
We can see that the fire reduced the habitat heterogeneity,\\ with a likely consequent decrease of biodiversity.  
\begin{figure}
    \centering
    \includegraphics[width=0.7\linewidth]{5plot.png}
\end{figure}
\end{frame}

\begin{frame}{Recovery}
\centering
Let's compare the area status before fire with the actual situation (August 2023). 
\begin{figure}
       \centering
       \includegraphics[width=0.8\linewidth]{6plot.png}
   \end{figure}   
\end{frame}

\begin{frame}{Recovery}
\centering
Let's do another comparison by calculating NDVI differences between August 2021 and August 2023.
\begin{figure}
     \centering
     \includegraphics[width=0.9\linewidth]{7plot.png}
 \end{figure} 
\end{frame}

\begin{frame}{Recovery}
\centering
We can clearly see that two years after the fire\\ secondary succession is restoring the environment.
\begin{figure}
     \centering
     \includegraphics[width=0.9\linewidth]{7plot.png}
 \end{figure} 
\end{frame}

%mettera posto la scala del grafico a sinistra

\section{Conclusion}

\begin{frame}{Resume}
\begin{itemize}
    \item Grandi differenze di \textbf{NDVI} tra stagione delle piogge e non.
    \bigskip
    \pause \item La classificazione evidenzia una \textbf{riduzione della vegetazione} del \textbf{85\%} tra marzo e settembre.
    \bigskip
    \pause \item L'aumento della vegetazione a marzo è \textbf{diffuso} su tutto il territorio.
    \bigskip
    \pause \item Settembre risulta avere un ambiente \textit{leggermente più eterogeneo}.
\end{itemize}
\end{frame}

\begin{frame}{The End}
\centering
\large \textbf{Grazie dell'attenzione!}
\end{frame}





\end{document}
